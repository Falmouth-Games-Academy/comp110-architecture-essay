\documentclass{scrartcl}

\usepackage[hidelinks]{hyperref}
\usepackage[none]{hyphenat}

\title{Essay Proposal}
\subtitle{COMP110 - Computer Architecture Essay}

\author{Samantha Wills}

\begin{document}

\maketitle

\section*{Topic}

My essay will be on
% Uncomment as appropriate:
%   convexity-based collision detection for a 2D game engine.
   procedural level generation for a 2D platform game.

\section*{Paper 1}
% This is an example! Replace the details with a paper relevant to your chosen topic.
\begin{description}
\item[Title:] Rhythm-based level generation for 2D platformers
\item[Citation:] \cite{Smith:2009}
\item[Abstract:] ``We present a rhythm-based method for the automatic generation of levels for 2D platformers, where the rhythm is that which the player feels with his hands while playing. Levels are created using a grammar-based method: first generating rhythms, then generating geometry based on those rhythms. Generation is constrained by a set of style parameters tweakable by a human designer. The approach also minimizes the amount of content that must be manually authored, instead relying on geometry components that are included in the level designer's tileset and a set of jump types. Our results show that this method produces an impressive variety of levels, all of which are fully playable.''
\item[Web link:] \url{http://dl.acm.org/citation.cfm?id=1536548}
\item[Full text link:] \url{https://users.soe.ucsc.edu/~ejw/papers/smith-platformer-generation-fdg2009.pdf}
\item[Comments:] This is a conference article that discusses the possibilities of creating playable levels with the incorporation of rhythm generation for geometry placed platforms.
\end{description}

\section*{Paper 2}
\begin{description}
\item[Title:] An approach to level design using procedural content generation and difficulty curves
\item[Citation:] \cite{diaz:2013}
\item[Abstract:] "Level design is an art which consists of creating the combination of challenge, competition, and interaction that players call fun and involves a careful and deliberate development of the game space. When working with procedural content generation, it is necessary to review how the game designer sets the change in difficulty throughout the different levels. In this paper we present a procedural level generator that can be used for different games and is based on a genetic algorithm. We define a fitness function that does not depend on the game or content type. This function calculates the difference between the difficulty curve defined by the designer and the difficulty curve calculated for the candidate content, so the best content is the one whose difficulty curve best fits the desired curve. To design our generator, we rely on the concept of flow, theories of fun and game design."
\item[Web link:] \url{http://ieeexplore.ieee.org/xpl/articleDetails.jsp?arnumber=6633640&queryText=procedural%20level%20generation%20for%202d%20platform&newsearch=true}
\item[Full text link:] \url{http://www.ijcsit.com/docs/Volume%206/vol6issue02/ijcsit2015060227.pdf}
\item[Comments:]  This article attempts to demonstrate how level creation should be generatored through functions that compare the intended difficultly by the desinger and the difficulty for the player content.
\end{description}

\section*{Paper 3}
\begin{description}
\item[Title:] PCG-based game design: creating Endless Web
\item[Citation:] \cite{Smith:2012}
\item[Abstract:] "This paper describes the creation of the game Endless Web, a 2D platforming game in which the player's actions determine the ongoing creation of the world she is exploring. Endless Web is an example of a PCG-based game: it uses procedural content generation (PCG) as a mechanic, and its PCG system, Launchpad, greatly influenced the aesthetics of the game. All of the player's strategies for the game revolve around the use of procedural content generation. Many design challenges were encountered in the design and creation of Endless Web, for both the game and modifications that had to be made to Launchpad. These challenges arise largely from a loss of fine-grained control over the player's experience; instead of being able to carefully craft each element the player can interact with, the designer must instead craft algorithms to produce a range of content the player might experience. In this paper we provide a definition of PCG-based game design and describe the challenges faced in creating a PCG-based game. We offer our solutions, which impacted both the game and the underlying level generator, and identify issues which may be particularly important as this area matures."
\item[Web link:] \url{http://dl.acm.org/citation.cfm?id=2282338.2282375&coll=DL&dl=GUIDE&CFID=729021823&CFTOKEN=29015746}
\item[Full text link:] \url{https://games.soe.ucsc.edu/sites/default/files/smith-fdg12.pdf}
\item[Comments:] This is article suggests the influence the player's actions have on an endless 2D platform game. It also reviews it's own generator aswell as stating the challenges of creating a PCG-based game.
\end{description}

\section*{Paper 4}
\begin{description}
\item[Title:] Polymorph: A Model for Dynamic Level Generation
\item[Citation:] \cite{Jennings:2010}
\item[Abstract:] "Players begin games at different skill levels and develop their skill at different rates—so that even the best-designed games are uninterestingly easy for some players and frustratingly difficult for others. A proposed answer to this challenge is Dynamic Difficulty Adjustment (DDA), a general category of approaches that alter games during play, in response to player performance. However, nearly all these techniques are focused on basic parameter tweaking,
while the difficulty of many games is connected to aspects that are more challenging to adjust dynamically, such as level design. Further, most DDA techniques are based on designer intuition, which may not reflect actual play patterns. Responding to these challenges, we have created Polymorph, which employs techniques from level generation and machine learning to understand level difficulty and player skill, dynamically constructing levels for a 2D platformer game with continually-appropriate challenge. We present the results of the user study on which Polymorph's model of level difficulty is based, as well as a discussion of the unique features of the model. We believe Polymorph creates a play experience that is unique because the changes are both personalized and structural, while also providing an example of a new application of machine learning to aid game design.
\item[Web link:] \url{http://aaai.org/ocs/index.php/AIIDE/AIIDE10/paper/view/2150}
\item[Full text link:] \url{https://www.aaai.org/ocs/index.php/AIIDE/AIIDE10/paper/.../2558}
\item[Comments:] This article suggests that the game level should be generatored with reference from the player's skill level and increased learning rate. It suggests a way to adjust the game level during play.
\end{description}

\section*{Paper 5}
\begin{description}
\item[Title:] {Patterns and procedural content generation: revisiting Mario in world 1 level 1}
\item[Citation:] \cite{Dahlskog:2012}
\item[Abstract:] "Procedural content generation and design patterns could potentially be combined in several different ways in game design. This paper discusses how to combine the two, using automatic platform game level design as an example. The paper also present work towards a pattern-based level generator for Super Mario Bros. (SMB), which is based on an analysis of the levels of the original SMB game where we found 23 different patterns."
\item[Web link:] \url{http://dl.acm.org/citation.cfm?id=2427116.2427117&coll=DL&dl=GUIDE&CFID=729021823&CFTOKEN=29015746}
\item[Full text link:] \url{julian.togelius.com/Dahlskog2012Patterns.pdf}
\item[Comments:] With reference to Mario world 1 level 1, this article is based on anaylsis and patterns found in SMB, and how PCG can be combined in different ways.
\end{description}

\bibliographystyle{ieeetr}
\bibliography{comp110_architecture}

\end{document}
