\documentclass{scrartcl}

\usepackage[hidelinks]{hyperref}
\usepackage[none]{hyphenat}

\title{Essay Proposal}
\subtitle{COMP110 - Computer Architecture Essay}

\author{Warwick New (1502903)}

\begin{document}

\maketitle

\section*{Topic}

My essay will be on
% Uncomment as appropriate:
%   convexity-based collision detection for a 2D game engine.
procedural level generation for a 2D platform game.

\section*{Paper 1}
% This is an example! Replace the details with a paper relevant to your chosen topic.
\begin{description}
\item[Title:] 
Human computation for procedural content generation in platform games
\item[Citation:] \cite{Reis}
\item[Abstract:] ``One of the major challenges in procedural content generation in computer games is to automatically evaluate whether the generated content has good quality. In this paper we describe a system which uses human computation to evaluate small portions of levels generated by an existing system for the game of Infinite Mario Bros. Several such evaluated portions are then combined into a full level of the game. The composition of the small portions into a full level is done by accounting for the human-annotated information and the mathematical model of tension arcs used in interactive drama and storytelling. We tested our system with human subjects and the results show that our approach is able to generate levels with better visual aesthetics and that are more enjoyable to play than other existing approaches.''
\item[Web link:] \url{http://ieeexplore.ieee.org/xpl/articleDetails.jsp?arnumber=7317906&newsearch=true&queryText=procedural\%20level\%20generation\%20for\%20a\%202D\%20platform\%20game}
\item[Full text link:] \url{http://www.eecs.harvard.edu/~gal/Papers/CIG_2015_submission_24.pdf}
\item[Comments:] This is a very recent article with many strong points and ideas on how procedurally generated levels can react to player feedback.
\end{description}

\section*{Paper 2}
\begin{description}
\item[Title:] Towards Automatic Personalized Content Generation for Platform Games
\item[Citation:] \cite{Shaker}
\item[Abstract:] ``In this paper, we show that personalized levels can be auto- matically
generated for platform games. We build on previ- ous work, where models
were derived that predicted player experience based on features of
level design and on playing styles. These models are constructed using
preference learn- ing, based on questionnaires administered to players
after playing different levels. The contributions of the current paper are (1) more accurate models based on a much larger data set; (2) a
mechanism for adapting level design parameters to given players and
playing style; (3) evaluation of this adap- tation mechanism using both
algorithmic and human players. The results indicate that the adaptation
mechanism effectively optimizes level design parameters for particular
players.''
\item[Web link:] \url{http://iris.ofai.at:7777/iris_db/index.php/publications/show/348}
\item[Full text link:] \url{http://www.aaai.org/ocs/index.php/AIIDE/AIIDE10/paper/viewFile/2135/2546}
\item[Comments:] This paper shows that a randomly generated level can be personalised for a player automatically. The paper has also undergone 5 years of peer review and has been cross referenced 42 times.
\end{description}

\section*{Paper 3}
\begin{description}
\item[Title:] A framework for analysis of 2D platformer levels
\item[Citation:] \cite{Smith}
\item[Abstract:] ``Levels are the space where a player explores the rules and mechanics of a game; as such, good level design is critical to the game design process. While there are many broad design principles, level design is inherently genre-specific due to the wide variety of rules and types of challenge found between genres. Determining genre-specific design principles requires an in-depth analysis of games within the genre. We present such an analysis for the 2D platformer genre, examining level components and structure with a view to better understanding their level design. We then use this analysis to present a model for platformer levels, specifically focusing on areas of challenge. Our framework provides a common vocabulary for these items and provides level designers with a method for thinking about elements of platformers and how to compose them to create interesting and challenging levels.''
\item[Web link:] \url{http://dl.acm.org/citation.cfm?doid=1401843.1401858}
\item[Full text link:] \url{https://users.soe.ucsc.edu/~ejw/papers/smith-sandbox-2008.pdf}
\item[Comments:] This document goes into detail on how to model power ups and how how movement aids can help a player in a level with momentum based game play. The paper has also been cited 14 times.
\end{description}

\section*{Paper 4}
\begin{description}
\item[Title:] 
Procedural level generation using occupancy-regulated extension
\item[Citation:] \cite{Mawhorter}
\item[Abstract:] ``Existing approaches to procedural level generation in 2D platformer games are, with some notable exceptions, procedures designed to do the work of a human game designer. They offer the usual benefits and disadvantages of AI applied to a cognitive task: they can work much faster than a human level designer, and are in some cases able to explore the design space automatically to find levels with desirable qualities. But they aren't able to capture the human creativity that produces the most interesting level designs, and they are usually very specific to their particular domain. This paper introduces occupancy-regulated extension (ORE), a general geometry assembly algorithm that supports human-design-based level authoring at arbitrary scales.''
\item[Web link:] \url{http://ieeexplore.ieee.org/xpl/articleDetails.jsp?arnumber=5593333&queryText=level\%20generation&newsearch=true}
\item[Full text link:] \url{https://games.soe.ucsc.edu/sites/default/files/cig10_043CP2_115.pdf}
\item[Comments:] This paper is great for the discussion on making cohesive randomly generated level designs for a general purpose plat former that fit an aesthetic requirement. The work has also been cited 7 times in the IEEE library.
\end{description}

\section*{Paper 5}
\begin{description}
\item[Title:] The Challenge of Automatic Level Generation for Platform Videogames Based on Stories and Quests
\item[Citation:] \cite{Mourato}
\item[Abstract:] ``In this article we bring the concepts of narrativism and ludology to automatic level generation for platform videogames. The initial motivation is to understand how this genre has been used as a storytelling medium. Based on a narrative theory of games, the differences among several titles have been identified. In addition, we propose a set of abstraction layers to describe the content of a quest-based story in the particular context of videogames. Regarding automatic level generation for platform videogames, we observed that the existing approaches are directed to lower abstraction concepts such as avatar movements without a particular context or meaning. This leads us to the challenge of automatically creating more contextualized levels rather than only a set of consistent and playable entertaining tasks. With that in mind, a set of higher level design patterns are presented and their potential usages are envisioned and discussed.''
\item[Web link:] \url{http://dl.acm.org/citation.cfm?id=2770015.2770041&coll=DL&dl=GUIDE}
\item[Full text link:] \url{http://comum.rcaap.pt/bitstream/123456789/6090/1/The\%20challenge\%20of\%20Automatic\%20Level\%20Generation\%20for\%20platform\%20videogames\%20base....pdf} (The ellipsis is necessary in the link)
\item[Comments:] This paper talks about generating levels around quest objectives and implementing stories into these levels.
\end{description}

\bibliographystyle{ieeetr}
\bibliography{comp110_architecture}

\end{document}
