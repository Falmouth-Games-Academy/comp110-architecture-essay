\documentclass{scrartcl}

\usepackage[hidelinks]{hyperref}
\usepackage[none]{hyphenat}

\title{Essay Proposal}
\subtitle{COMP110 - Computer Architecture Essay}

\author{Dean Harland (DH185421)}

\begin{document}
	
	\maketitle
	
	\section*{Topic}
	
	My essay will be on procedural level generation for a 2D platform game.
	
	
	\section*{Paper 1}
	% This is an example! Replace the details with a paper relevant to your chosen topic.
	\begin{description}
		\item[Title:] A Framework for Analysis of 2D Platformer Levels
		\item[Citation:] \cite{tricky}
		\item[Abstract:] ``Levels are the space where a player explores the rules and mechanics
		of a game; as such, good level design is critical to the game design
		process. While there are many broad design principles, level
		design is inherently genre-specific due to the wide variety of rules
		and types of challenge found between genres. Determining genrespecific
		design principles requires an in-depth analysis of games
		within the genre. We present such an analysis for the 2D platformer
		genre, examining level components and structure with a view to
		better understanding their level design. We then use this analysis
		to present a model for platformer levels, specifically focusing on
		areas of challenge. Our framework provides a common vocabulary
		for these items and provides level designers with a method for
		thinking about elements of platformers and how to compose them
		to create interesting and challenging levels.''
		\item[Web link:] \url{http://citeseerx.ist.psu.edu/viewdoc/download?doi=10.1.1.494.4080&rep=rep1&type=pdf}
		\item[Full text link:] \url{http://citeseerx.ist.psu.edu/viewdoc/download?doi=10.1.1.494.4080&rep=rep1&type=pdf}
		\item[Comments:] This is an informative article on the more ticker parts if the level design, I believe this to be relevant to the point as just finding techniques to finish the level does not contribute to the players enjoyment of the game and finding harder routes is a way of making the player enjoy it more. cited by over 40 people although it is not many it still shows that a few dozen people have. This was found through the internet and key word searches.
	\end{description}
	
	\section*{Paper 2}
	\begin{description}
		\item[Title:] Empirical evaluation of procedural level
		generators for 2D platform games
		\item[Citation:] \cite{empirical}
		\item[Abstract:] ''Context. Procedural content generation (PCG) refers to algorithmical creation of game content (e.g.
		levels, maps, characters). Since PCG generators are able to produce huge amounts of game content, it
		becomes impractical for humans to evaluate them manually. Thus it is desirable to automate the
		process of evaluation.
		Objectives. This work presents an automatic method for evaluation of procedural level generators for
		2D platform games. The method was used for comparative evaluation of four procedural level
		generators developed within the research community.
		Methods. The evaluation method relies on simulation of the human player's behaviour in a 2D
		platform game environment. It is made up of three components: (1) a 2D platform game Infinite Mario
		Bros with levels generated by the compared generators, (2) a human-like bot and (3) quantitative
		models of player experience. The bot plays the levels and collects the data which are input to the
		models. The generators are evaluated based on the values output by the models. A method based on
		the simple moving average (SMA) is suggested for testing if the number of performed simulations is
		sufficient.
		Results. The bot played all 6000 evaluated levels in less than ten minutes. The method based on the
		SMA showed that the number of simulations was sufficiently large.
		Conclusions. It has been shown that the automatic method is much more efficient than the traditional
		evaluation made by humans while being consistent with human assessments.''
		\item[Web link:] \url {http://www.diva-portal.org/smash/get/diva2:831319/FULLTEXT01.pdf}
		\item[Full text link:] \url {http://www.diva-portal.org/smash/get/diva2:831319/FULLTEXT01.pdf}
		\item[Comments:] Cited by over 300 people this is a more credible source. I believe this is relevant because it introduces the PCG and how it can create thousands of levels and then have a bot run through these levels and collects data and evaluates the results. This was found through the internet and key word searches.
	\end{description}
	
	\section*{Paper 3}
	\begin{description}
		\item[Title:] Dynamically adjusting game-play in 2D Platformers using
		Procedural Level Generation
		\item[Citation:] \cite{dynamic}
		\item[Abstract:] ''The rapid growth of the entertainment industry has presented the requirement for more efficient
		development of computerized games. Importantly, the diversity of audiences that participate in
		playing games has called for the development of new technologies that allow games to address
		users with differing levels of skills and preferences. This research presents a systematic study that
		explored the concept of dynamic difficulty using procedural level generation with interactive
		evolutionary computation. Additionally, the design, development and trial of computerized agents
		the play game levels in the place of a human player is detailed. The work presented in this thesis
		provides a solution to the rapid growth of the entertainment industry whilst providing a more
		effective means for developing computerized games''
		\item[Web link:] \url{http://ro.ecu.edu.au/cgi/viewcontent.cgi?article=1095&context=theses_hons}
		\item[Full text link:] \url{http://ro.ecu.edu.au/cgi/viewcontent.cgi?article=1095&context=theses_hons}
		\item[Comments:] By just reading the abstract alone I found this one to be greatly interesting and relative to the point at hand. This was found through the internet and key word searches.
	\end{description}
	
	\section*{Paper 4}
	\begin{description}
		\item[Title:] Rhythm-Based Level Generation for 2D Platformers
		\item[Citation:] \cite{Rhytem}
		\item[Abstract:] ''We present a rhythm-based method for the automatic generation
		of levels for 2D platformers, where the rhythm is that which the
		player feels with his hands while playing. Levels are created using
		a grammar-based method: first generating rhythms, then
		generating geometry based on those rhythms. Generation is
		constrained by a set of style parameters tweakable by a human
		designer. The approach also minimizes the amount of content that
		must be manually authored, instead relying on geometry
		components that are included in the level designer’s tileset and a
		set of jump types. Our results show that this method produces an
		impressive variety of levels, all of which are fully playable. ''
		
		\item[Web link:] \url{https://users.soe.ucsc.edu/~ejw/papers/smith-platformer-generation-fdg2009.pdf}
		\item[Full text link:] \url{https://users.soe.ucsc.edu/~ejw/papers/smith-platformer-generation-fdg2009.pdf}
		\item[Comments:] Rhythm-Based Level Generation for 2D Platformers is an interesting concept on 2D platform creation and how it produces a variety of levels, it is also cited by over 70 people. This was found through the internet and key word searches.
	\end{description}
	
	\section*{Paper 5}
	\begin{description}
		\item[Title:] Level Generation System for Platform Games
		Based on a Reinforcement Learning Approach 
		\item[Citation:] \cite{learning}
		\item[Abstract:]''Automated level generation is a topic of much controversy in the video games
		industry that divides the opinions strongly for and against it. The current situation
		is that generation techniques are in common use only in a few specific genres of
		video games for reasons that can be principal, practical and in some cases entirely
		subjective. At the same time there is a widespread tendency for game worlds to
		become larger and more repayable. Manual level design becomes an expensive
		task and the necessity for automated productivity tools is growing.
		In this project I focus my efforts on the creation of a level generation system for a
		genre of video games that is very conservative in its use of these techniques.
		Automated generation for platform video games also presents a technological
		challenge because there are hard restrictions on what constitutes a valid level.
		The intuition for choosing reinforcement learning as a generative approach is based
		on the structure of platform game levels, which lends itself to a natural
		representation as a sequential decision making process.''
		\item[Web link:]\url{https://www.inf.ed.ac.uk/publications/thesis/online/IM090699.pdf}
		\item[Full text link:] \url{https://www.inf.ed.ac.uk/publications/thesis/online/IM090699.pdf}
		\item[Comments:] This article is relative as it talks about how generic level creation can lead to stale levels and repetition, where as this talks about decision making processes. This was found through the internet and key word searches.
	\end{description}
	
\end{document}
