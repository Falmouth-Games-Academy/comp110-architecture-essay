\documentclass{scrartcl}

\usepackage[hidelinks]{hyperref}
\usepackage[none]{hyphenat}

\title{Essay Proposal}
\subtitle{COMP110 - Computer Architecture Essay}

\author{Harriet Moore}

\begin{document}

\maketitle

\section*{Topic}

My essay will be on procedural level generation for a 2D platform game.

\section*{Paper 1}
\begin{description}
\item[Title:] Launchpad: A Rhythm-Based Level Generator for 2-D Platformers
\item[Citation:] \cite{smith:launchpad}
\item[Abstract:] ``Launchpad is an autonomous level generator that is based on a formal model of 2-D platformer level design. Levels are built out of small segments called “rhythm groups,” which are generated using a two-tiered, grammar-based approach. These segments are pieced together into complete levels that are then rated according to a set of design heuristics. Generation can be controlled using a set of parameters that influence the level pacing and geometry. The approach minimizes the amount of content that must be manually authored: instead of piecing together large segments of a level, Launchpad uses base components that are commonly found in a number of 2-D platformers. Launchpad produces an impressive variety of levels which are all guaranteed to be playable.''
\item[Web link:] \url{http://ieeexplore.ieee.org/xpl/articleDetails.jsp?arnumber=5648340}
\item[Full text link:] \url{https://users.soe.ucsc.edu/~ejw/papers/launchpad-smith-tciaig-2011.pdf}
\item[Comments:] I found this article on the IEEE Xplore database. It is relevant to the essay as it is about a tool for procedurally generating levels for 2D platform games based on the flow of the level. It is published by IEEE, so it is from a reputable publisher.
\end{description}

\section*{Paper 2}
\begin{description}
\item[Title:] Procedural level generation using occupancy-regulated extension
\item[Citation:] \cite{mawhorter:occupancy}
\item[Abstract:] Existing approaches to procedural level generation in 2D platformer games are, with some notable exceptions, procedures designed to do the work of a human game designer. They offer the usual benefits and disadvantages of AI applied to a cognitive task: they can work much faster than a human level designer, and are in some cases able to explore the design space automatically to find levels with desirable qualities. But they aren't able to capture the human creativity that produces the most interesting level designs, and they are usually very specific to their particular domain. This paper introduces occupancy-regulated extension (ORE), a general geometry assembly algorithm that supports human-design-based level authoring at arbitrary scales.
\item[Web link:] \url{http://ieeexplore.ieee.org/xpl/articleDetails.jsp?arnumber=5593333}
\item[Full text link:] \url{https://games.soe.ucsc.edu/sites/default/files/cig10_043CP2_115.pdf}
\item[Comments:] I found this paper in the IEEE Xplore database. It is published by IEEE so is from a reputable publisher. It is relevant to the essay as it describes a technique for procedural level generation in platform games that requires more input from a human than some of the other techniques.
\end{description}

\section*{Paper 3}
\begin{description}
\item[Title:] Automatic Level Generation for Platform Videogames Using Genetic Algorithms
\item[Citation:] \cite{mourato:genetic}
\item[Abstract:] In this document we present an investigation on automatically generating levels for platform videogames. Common approaches for this problem are rhythm based, where input patterns are transformed in a valid geometry, and chunk based, where samples are humanly created and automatically assembled like a puzzle. The proposal hereby presented is to explore this challenge with the usage of Genetic Algorithms, facing it as a search problem, in order to achieve higher expressivity and less linearity than in rhythm based approach and without requiring human creation as it happens with the chunk based approach. With simple heuristics the system is able to generate playable levels in a small amount of time (one level is created in less than a minute) and with considerable diversity, as our results show.
\item[Web link:] \url{http://dl.acm.org/citation.cfm?id=2071423.2071433}
\item[Full text link:] \url{}
\item[Comments:] I found this paper in the ACM Digital Library. It presents an approach to procedural content generation for platform games that is different to those presented in the above papers \cite{smith:launchpad, mawhorter:occupancy}. It is published by ACM, so it is from a reputable publisher.
\end{description}

\section*{Paper 4}
\begin{description}
\item[Title:] A procedural procedural level generator generator
\item[Citation:] \cite{kerssemakers:procedural}
\item[Abstract:] Procedural content generation (PCG) is concerned with automatically generating game content, such as levels, rules, textures and items. But could the content generator itself be seen as content, and thus generated automatically? This would be very useful if one wanted to avoid writing a content generator for a new game, or if one wanted to create a content generator that generates an arbitrary amount of content with a particular style or theme. In this paper, we present a procedural procedural level generator generator for Super Mario Bros. It is an interactive evolutionary algorithm that evolves agent-based level generators. The human user makes the aesthetic judgment on what generators to prefer, based on several views of the generated levels including a possibility to play them, and a simulation-based estimate of the playability of the levels. We investigate the characteristics of the generated levels, and to what extent there is similarity or dissimilarity between levels and between generators.
\item[Web link:] \url{http://ieeexplore.ieee.org/xpl/articleDetails.jsp?arnumber=6374174}
\item[Full text link:] \url{}
\item[Comments:] I found this paper in the IEEE Xplore database. It is co-authored by a well known researcher in the field and published by IEEE. It is relevant to the essay as the work is a procedural content generator that generates procedural level generators, which may lead to a wider variety of levels being able to be produced.
\end{description}

\section*{Paper 5}
\begin{description}
\item[Title:] A Generic Approach to Challenge Modeling for the Procedural Creation of Video Game Levels
\item[Citation:] \cite{sorenson:generic}
\item[Abstract:] This paper presents an approach to automatic video game level design consisting of a computational model of player enjoyment and a generative system based on evolutionary computing. The model estimates the entertainment value of game levels according to the presence of “rhythm groups,” which are defined as alternating periods of high and low challenge. The generative system represents a novel combination of genetic algorithms (GAs) and constraint satisfaction (CS) methods and uses the model as a fitness function for the generation of fun levels for two different games. This top-down approach improves upon typical bottom-up techniques in providing semantically meaningful parameters such as difficulty and player skill, in giving human designers considerable control over the output of the generative system, and in offering the ability to create levels for different types of games.
\item[Web link:] \url{ieeexplore.ieee.org/xpl/articleDetails.jsp?arnumber=5940995}
\item[Full text link:] \url{http://ivizlab.sfu.ca/media/2011_TransactionsAI.pdf}
\item[Comments:] I found an earlier work from these researchers cited in a few of the other papers I came across. I then searched for the author's name and found this paper, which appears to have expanded on the earlier work. It was published by IEEE so is from a reputable publisher.
\end{description}

\bibliographystyle{ieeetr}
\bibliography{comp110_architecture}

\end{document}
