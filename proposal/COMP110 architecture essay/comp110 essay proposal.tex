\documentclass{article}

\title{Essay Proposal - COMP110 Computer Architecture Essay}

\author{TE182503}

\begin{document}

\maketitle

\section*{Topic}
My essaye will be on Convexity-based collision detection for a 2D game engine.

\section*{Paper 1}

\begin{description}
\item[title:] The Construction of a Predictive Collision 2D Game Engine
\item[Citation:]\cite{<mulley2013construction>} 
\item[Abstract:]"This paper discusses the construction of a game engine which is based around the principle of discrete event simulation. This work is interesting as it uses a predictive time of collision rather than a frame based approach. The key design decisions made and the tools used during the construction of the predictive game engine (PGE) are described. The objects modelled in the game engine are rigid circles and polyhedra which may have an orbit rotational velocity, a positional velocity and acceleration. Equations calculating the time of next collision between two circles, a line and a circle and two lines are given. If orbit rotation is excluded, the two circles expansion is simple as the output from maxima is only twelve lines, conversely when orbit rotation is included the output exceeds 835 lines. The input formula to maxima used for detecting the time of collisions between moving polyhedra are presented alongside a tool to automatically import the expanded formula into program code."
\item[web link:] \url{http://ieeexplore.ieee.org/xpl/login.jsp?tp=&arnumber=7004920&url=http%3A%2F%2Fieeexplore.ieee.org%2Fxpls%2Fabs_all.jsp%3Farnumber%3D7004920}
\item[Full text link:]\url{http://floppsie.comp.glam.ac.uk/Papers/paper21/2d-game-engine-mulley.pdf}
\item[Comments:] i found this paper which is about predictive collision in 2d game engines. This paper looks into the construction of an engine which uses predictive collision over a frame based approach which means collisons are calculated in advance.
\end{description}

\section*{Paper 2}

\begin{description}
\item[Title:] An Improved Algorithm of Collision Detection in 2D Grapple Games
\item[Citation:]\cite{<guo2010improved>}
\item[Abstract:]"The goal of collision detection is to automatically report interference between two or more geometric objects in static and dynamic environments. We introduce a new approach to the problem of collision detection in 2D Grapple Games. The objects for collision detection are bounded by the axis-aligned rectangle and circle for a tightly fitting the objects' shapes. To detect the collision exactly, an improved algorithm is presented. We set coordinate values of rectangular center and four vertexes and use the distance from a Point to a line to see whether the object's vertexes are enclosed in the other object. if it is true, a collision occurs. Collision detection is part of the handling collision and its result is a Boolean judgment about the collision of two or more objects colliding."
\item[web link: ] \url{http://ieeexplore.ieee.org/xpl/login.jsp?tp=&arnumber=5453586&url=http%3A%2F%2Fieeexplore.ieee.org%2Fxpls%2Fabs_all.jsp%3Farnumber%3D5453586}
\item[Full text link:]
\item[Comments:]This paper is about automaticaly detecting inteference between two objects it provides an algorithm which detects collisions exactly an sees if objects vertexes are enclosed.
\end{description}

\section*{Paper 3} 

\begin{description}
\item[Title:] Distance computation using axis aligned bounding box (AABB) parallel distribution of dynamic origin point
\item[Citation:]\cite{<sulaiman2013distance>}
\item[Abstarct:]"Performing accurate and precise collision detection method between objects in virtual environment application such as computer games and medical simulation is important in computer graphics research and development. Given pair of objects that near colliding, numerous mechanic has been developed by researchers in order to minimize computation time and increase accuracy of the detection. However, most of these techniques required a lot of computational cost, extra processing power and complex algebraic equations just to solve distance between near colliding objects. In this paper, we described an alternate technique, which is a theoretical framework of novel technique in order to find the optimum closest distance between two or more convex polyhedral in virtual environment application. Given pair of near colliding objects, we proposed an easy to implement mechanism using dynamic origin point by creating inner and middle Axis Aligned Bounding-Box just to find closest distance between objects. We believed that the technique is suitable to be used in any game engine tools for computer games and medical simulation."
\item[web link:] \url{http://ieeexplore.ieee.org/xpl/login.jsp?tp=&arnumber=6575955&url=http%3A%2F%2Fieeexplore.ieee.org%2Fxpls%2Fabs_all.jsp%3Farnumber%3D6575955}
\item[Full text link:]
\item[Comments:]this paper introduces a new technique which looks for the closest distance between two objects. this can make colision detection more effective.
\end{description}

\section*{Paper 4}

\begin{description}
\item[Title:] Interactive Mesostructures withVolumetric Collisions
\item[Citation:]\cite{<nykl2014interactive>}
\item[Abstarct:]"This paper presents a technique for interactively colliding with and deforming mesostructures at a per-texel level. It is compatible with a broad range of existing mesostructure rendering techniques including both safe and unsafe ray-height field intersection algorithms. This technique is able to replace traditional 3D geometrical deformations (vertex-based) with 2D image space operations (pixel-based) that are parallelized on a GPU without CPU-GPU data shuffling and integrates well with existing physics engines. Additionally, surface and material properties may be specified at a per-texel level enabling a mesostructure to possess varying attributes intrinsic to its surface and collision behavior. Furthermore, this approach may replace traditional decals with image-based operations that naturally accumulate deformations without inserting any new geometry. This technique provides a simple and efficient way to make almost every surface in a virtual world responsive to user actions and events. It requires no preprocessing time and storage requirements of one additional texture or less. The algorithm uses existing inverse displacement map algorithms as well as existing physics engines and can be easily incorporated into new or existing game pipelines."
\item[web link:]\url{http://ieeexplore.ieee.org/xpl/articleDetails.jsp?arnumber=6799298}
\item[Full text link:]
\item[Comments:]This paper looks into a new technique for interactively coliding with and deforming mesostructures at a par text level. it is a new algorithm which can be used with existing physics engines and new games so it can be a more effective physics engine for games.it allows for accumulating deformations without the need for new geomatry. 
\end{description}

\section*{Paper 5}

\begin{description}
\item[Title:] Collision course by transformation of coordinates and plane decomposition
\item[Citation:]\cite{<bendjilali2009collision>}
\item[Abstarct:]"This paper deals with the problem of collision course checking in a dynamic environment for mobile robotics applications. Our method is based on the relative kinematic equations between moving objects. These kinematic equations are written under polar form. A transformation of coordinates is derived. Under this transformation, collision between two moving objects is reduced to collision between a stationary object and a virtual moving object. In addition to the direct collision course, we define the indirect collision course, which is more critical and difficult to detect. Under this formulation, the collision course problem is simplified, and complex scenarios are reduced to simple scenarios. In three-dimensional (3D) settings, the working space is decomposed into two planes: the horizontal plane and the vertical plane. The collision course detection in 3D is studied in the vertical and horizontal planes using 2D techniques. This formulation brings important simplifications to the collision course detection problem even in the most critical and difficult scenarios. An extensive simulation is used to illustrate the method in 2D and 3D working spaces."
\item[web link:] \url{http://dl.acm.org/citation.cfm?id=1552118}
\item[Full text link:]
\item[Comments:]This paper looks into kikemtic equations. It makes collision detection simpler, it does this by making the complex scenarios simple ones.
\end{description}

\bibliographystyle{ieeetr}
\bibliography{architecture_references}

\end{document}