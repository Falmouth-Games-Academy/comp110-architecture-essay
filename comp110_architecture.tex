% Please do not change the document class
\documentclass{scrartcl}

% Please do not change these packages
\usepackage[hidelinks]{hyperref}
\usepackage[none]{hyphenat}
\usepackage{setspace}
\doublespace

% You may add additional packages here
\usepackage{amsmath}

% Please include a clear, concise, and descriptive title
\title{An Analysis of the Variability, Reliability, and Control Provided by Level Generation Techniques for 2D Platform Games}

% Please do not change the subtitle
\subtitle{COMP110 - Computer Architecture Essay}

% Please put your student number in the author field
\author{1507290}

\begin{document}

\maketitle

\abstract{There are a number of diverse approaches to level generation for 2D platform games, each placing emphasis on a different aspect of level generation. Chunk-based approaches such as Occupancy Regulated Extension tend to favour control and variation, whilst rhythm-based approaches such as that demonstrated by Launchpad favour playability and enjoyability. There are also evolutionary approaches that use genetic algorithms, which when combined with constraint satisfaction, as demonstrated by Sorenson et al., provide a high level of control and reliability.}

\section{Introduction}
Procedural content generation for 2D platform games is a relatively new field in comparison to games of other genres~\cite{compton:platform}. Many diverse approaches to this problem have been developed, each having advantages and disadvantages.~\cite{horn:comparative}. There are chunk-based approaches which rely on hand-designed chunks of level; rhythm-based models that generate a level based on sequences of player actions; and evolutionary approaches that use genetic algorithms. It could be argued that the most desirable level generator would be the one that requires the least human input yet offers high amounts of control and is able to generate a wide variety of fun and playable levels consistently. 

\section{Comparison}

\subsection{Overview}
Smith et al.~\cite{smith:launchpad} base their level generation tool, Launchpad, on the concept of rhythm. They define rhythm as being the ideal sequence and timing of button presses that the player must execute; for example, a series of jumps. They use a two-tiered approach, generating rhythm separately and ensuring it is always present regardless of the level geometry. Rhythm groups are generated first, before being translated into geometry using design grammars. 

Sorenson et al.~\cite{sorenson:generic} are also inspired by rhythm with their technique, but instead define rhythm as alternating periods of high and low difficulty. They use a genetic algorithm to generate levels. The quality of a level is determined by a fitness function, and levels that have a high fitness value are able to pass their genetic information on to future generations, resulting in levels gradually being improved. They also use constraint satisfaction to ensure playability and provide the designer additional control. Although other approaches using evolutionary algorithms have been developed~\cite{mourato:genetic}, Sorenson et al.'s offers more direct control. They designed the algorithm to be general and able to apply to any genre of game.

Mawhorter and Mateas~\cite{mawhorter:occupancy} take a different approach to level generation with Occupancy-Regulated Extension (ORE). Instead of levels being generated from individual components, ORE uses pre-authored chunks of level and stitches them together based on anchor points that represent places the player can occupy. The system was designed to prioritise variability, with no strict measures in place to ensure playability, unlike the other two algorithms.

\subsection{Human Input and Control}
Launchpad is designed to require minimal effort whilst offering high levels of control. It allows the rhythm, the path of the level, and the desired number of each component to be specified. A number of levels are generated and assessed before a pool of the best matches is presented to the designer. In practice, this means that Launchpad may be better for aiding the design of levels, rather than generating them in-game. Even if the closest matching level could automatically be chosen, generating a large pool of levels would make the process take longer with no visible results.
%The approaches of Smith et al. and Sorenson et al. were designed to require minimal human effort whilst offering high levels of control. Launchpad allows the designer to specify the rhythm, the path that the level should follow, and the number of each component. The system generates a number of levels and assesses how close they are to the requirements, presenting the designer with a pool of the best fitting levels to choose from. In practice, this means that it may be a better system for aiding the design of levels, rather than generating them in-game. Although, the level with the highest fit could be automatically chosen.

Sorenson et al.'s approach also allows the number of each design element to be restricted. Additionally, the designer can adjust the difficulty of the generated levels by changing one parameter. If desired, part of the level can be hand-designed and the rest of the level generated around it. This means that level profiles for each part of the game could be specified, allowing control over progression whilst maintaining variation. If the player repeatedly loses, perhaps the difficulty parameter could be reduced for the next level generated. It can be noted, however, that the time taken is too long to generate levels in-game. Although, it is possible to reduce the time taken to find a satisfactory level by seeding the initial population with levels of high quality.
%Possibly mention challenge modelling - adjusting difficulty based on player skill

%On the other hand, Sorenson and Pasquier state that it may not be practical to generate levels as the player is playing, although  well-designed seed levels may make it possible to quickly generate suitable levels \cite{shaker:mario}. Of course, this would require the one-time effort of spending time designing seed levels.

Of the three methods, ORE requires the most human effort. Level chunks must be hand-authored, meaning that the output is greatly influenced by the quality of the chunks. Additionally, in order to achieve the desired result, an understanding of how the algorithm works is required. One advantage of hand designing chunks, however, is that chunks can be of any nature, allowing more complex structures to be generated.
%While the designer can influence the output explicitly in this way, they can also influence the output by assigning tags to chunks and specify the types of chunks to be used.


\subsection{Flexibility and Variability}
%The level of required and optional human input can have an impact on the flexibility and variation between levels able to be generated. Additionally, the design of the level generator itself greatly influences this aspect.

Smith et al. claim that their two-tiered approach allows to produce a wide variety of levels. That is, levels with the same rhythm can have many variations. Certain patterns and tendencies can be observed in the levels generated by Launchpad, however. The levels tend to be very linear, and they have little pattern variation~\cite{horn:comparative}. Although the levels are different, they may begin to feel repetitive after extended play time. Additionally, the emphasis on rhythm means that the generated levels are only appropriate for dexterity based platform games.

This is also an issue with Sorenson et al.'s algorithm, as the model of fun used in the fitness function is based around challenge and difficulty with respect to skill. On the other hand, there is the scope to adjust the fitness function. If required, a fitness function based on a different model of fun could be developed, to allow levels suitable for other types of platform games to be generated. Furthermore, their implementation of rhythm is looser, allowing levels to be less linear and repetitive.

ORE could potentially generate levels appropriate for other types of platform games, such as those that focus on exploration. This would be possible as chunks can be designed with any desired result in mind, whilst the algorithm does not need adjusting since the playability of levels is not assessed. Additionally, much variation in patterns on the generated levels can be observed~\cite{horn:comparative}, which could have a positive impact the fun and replayability of levels generated by ORE.


\subsection{Playability and Design}
Consequently, this means that ORE may not always produce playable levels (that is, levels able to be completed). This reduces the possible applications of ORE, as it would be risky to use it to generate levels in-game in case the player is presented with an unbeatable level. One option is to use an additional algorithm to determine whether the level is playable before presenting it to the player, but that would use additional resources and extend the time taken.

The other two approaches have comprehensive algorithms in place to ensure playability. Launchpad has a physics engine that knows how fast the player moves and how high they can jump, which is used when generating the level geometry to ensure that the resulting level is possible to complete.
Sorenson et al. specify the requirements for playability in the form of constraints, meaning that unplayable levels are repaired using constraint satisfaction before being returned to the population of acceptable levels. 

In addition to having measures in place to ensure playability, both ensure that the levels are fun. Smith et al.~\cite{smith:rhythm} and Sorenson et al.'s definitions of fun are inspired by the concept of 'flow' as described by Csikszentmihalyi~\cite{csik:flow}. The fitness function used in Sorenson et al.'s genetic algorithm is based on their comprehensive model of fun~\cite{sorenson:fun}, ensuring that the population evolves towards levels with a higher fitness value. Smith et al.'s definition of fun is defined in the algorithm that generates the rhythm groups. Although 'fun' is subjective, these two algorithms ensure that at least some form of theoretical 'fun' will be present in the game.

ORE is not based on any model of fun. The enjoyability of the resulting level depends both on how well the chunks have been designed and the implementation of the chunk selection algorithm. Again, this puts a lot of emphasis on the designer. With chunks designed by the authors of the algorithm, ORE came fourth in the \textit{2010 Mario AI Championship}~\cite{shaker:mario}, whilst Sorenson et al. managed to place third, despite their approach being generalised.


\section{Conclusion}
Whilst ORE provides great control over the design of a generated level, Launchpad and Sorenson et al.'s approach provide greater flexibility and reliability with little effort. Sorenson et al.'s approach perhaps provides greater control and flexibility than launchpad; parameters are able to easily be altered and level sections can be designed by hand, in addition to providing control over component numbers. They have comprehensively modeled the concept of 'fun' and ensured that all levels generated comply with it. Its success is demonstrated by their reasonable placing in the \textit{2010 Mario AI Championship}~\cite{shaker:mario}, despite being a general approach. Overall, Sorenson et al.'s approach reliably produces fun and playable levels, whilst providing a high amount of control whilst requiring little effort.

\bibliographystyle{ieeetr}
\bibliography{comp110_architecture}

\end{document}
